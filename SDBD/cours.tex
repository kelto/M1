\documentclass[10pt,a4paper]{article}
\usepackage[utf8]{inputenc}
\usepackage[french]{babel}
\usepackage[T1]{fontenc}
\usepackage{amsmath}
\usepackage{amsfonts}
\usepackage{amssymb}



\title{Système de Base de Données}
\date{}

\begin{document}
\maketitle

\part{Intro et rappel BD et SGBD}
\part{Evaluer et optimiser une requete relationnel}
\section{introduction}
\begin{enumerate}
\item[évaluer :] compiler
\item[analyse :] plan d'exécution (arborescente cf image 1)
\item[sémantique :] le sens (typage), typage retrouvé dans la métabase ainsi que les contraintes, etc ...
\end{enumerate}

optimalité = minimiser le temps de réponse

(arbre 2)
arbre linéaire gauche (((R1$\Join$R2)$\Join$R3)$\Join$R4)

(arbre 3)
arbre linéaire droit (R1$\Join$(R2$\Join$(R3$\Join$R4)))

(arbre 4)
arbre ramifié ((R1$\Join$R2)$\Join$(R3$\Join$R4))
\section{Comment implémenter les relation de base}

\begin{eqnarray*}
	\mid page \mid = 4Ko = 35 tuples net (40 tuples brut)
\end{eqnarray*}

\subsection{organisation hachée (séquence aléatoire)}
4 moyens :
\begin{enumerate}
\item séquentiel
\item séquentiel indexé
\item relatif
\item table de hachage
\end{enumerate}
\part{Introduction aux transactions}
\section{Introduction}


\end{document}