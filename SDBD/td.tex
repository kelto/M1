\documentclass[10pt,a4paper]{article}
\usepackage[utf8]{inputenc}
\usepackage[french]{babel}
\usepackage[T1]{fontenc}
\usepackage{amsmath}
\usepackage{amsfonts}
\usepackage{amssymb}
\usepackage{listings<`3`>}
\usepackage{xcolor}
\definecolor{gris1}{gray}{0.40}
\definecolor{gris2}{gray}{0.55}
\definecolor{gris3}{gray}{0.65}
\definecolor{gris4}{gray}{0.50}
\definecolor{vert}{rgb}{0,0.4,0}
\definecolor{violet}{rgb}{0.65, 0.2, 0.65}
\definecolor{bleu1}{rgb}{0,0,0.8}
\definecolor{bleu2}{rgb}{0,0.2,0.6}
\definecolor{bleu3}{rgb}{0,0.2,0.2}
\definecolor{rouge}{HTML}{F93928}


\lstdefinelanguage{algo}{%
   morekeywords={%
    %%% couleur 1
		importer, programme, glossaire, fonction, procedure, constante, type, 
	%%% IMPORT & Co.
		si, sinon, switch, alors, fin, tantque, debut, faire, lorsque, fin lorsque, 
		declenche, declencher, enregistrement, tableau, retourne, retourner, =, pour, a,
		/=, <, >, traite,exception, 
	%%% types 
		Entier, Reel, Booleen, Caractere, Réél, Booléen, Caractère,
	%%% types 
		entree, maj, sortie,entrée,
	%%% types 
		et, ou, non,
		Debut, début, debut, fin, Fin, faux, vrai, Faux, Vrai
	},
  sensitive=true,
  morecomment=[l]{--},
  morestring=[b]',
}

\lstset{language=algo,
    %%% BOUCLE, TEST & Co.
      emph={importer, programme, glossaire, fonction, procedure, constante, type},
      emphstyle=\color{bleu2},
    %%% IMPORT & Co.  
	emph={[2]
		si, sinon, switch, alors, fin , tantque, debut, faire, lorsque, fin lorsque, 
		declencher, retourner, et, ou, non,enregistrement, retourner, retourne, 
		tableau, /=, <, =, >, traite,exception, pour, a
	},
      emphstyle=[2]\color{bleu1},
    %%% FONCTIONS NUMERIQUES
		emph={[3]Vrai, vrai, Faux, faux},
      emphstyle=[3]\color{red},
    %%% FONCTIONS NUMERIQUES
      emph={[4]entree, maj, sortie, entrée},	
      emphstyle=[4]\color{gris1},
}


\lstset{ % general style for listings 
   numbers=left 
   , literate={é}{{\'e}}1 {è}{{\`e}}1 {à}{{\`a}}1 {ê}{{\^e}}1 {É}{{\'E}}1 {ô}{{\^o}}1 {€}{{\euro}}1{°}{{$^{\circ}$}}1 {ç}{ {c}}1 {ù}{u}1
	, extendedchars=\true
   , tabsize=2 
   , stepnumber=2
   , frame=l
   , framerule=1.1pt
   , linewidth=510px
   , breaklines=true 
   , basicstyle=\footnotesize\ttfamily 
   , numberstyle=\tiny\ttfamily 
   , framexleftmargin=0mm 
   , xleftmargin=0mm 
   , captionpos=b 
	, keywordstyle=\color{bleu2}
	, commentstyle=\color{vert}
	, stringstyle=\color{rouge}
	, showstringspaces=false
	, extendedchars=true
	, mathescape=true
	, prebreak=\raisebox{0ex}[0ex][0ex]
		   {\ensuremath{\hookleftarrow}}
	,breakatwhitespace=true
} 
%	\lstlistoflistings
%	\addcontentsline{toc}{part}{List of code examples}

\title{SDBD\\TD}
\date{}
\author{Charles Fallourd}
\begin{document}
\maketitle
\part*{Transaction}
\section*{Cohérence}
R(X); lecture de X \\
W(X); écriture de X \\
a; Abort \\
C; Commit \\
$R_i(X) \backsim T_i \cdot R(X)$ \\
\begin{enumerate}
\item $R_1(X); W_1(X); R_2(X); W_2(X); a_1; C_2$
\item $R_1(X); W_1(X); R_2(Y); W_2(Y); a_1; C_2$
\item $R_1(X); R_2(X); R_2(Y); W_2(Y); R_1(Z); a; C_2$
\item $R_1(X); R_2(X); W_2(X); W_1(X); C_1; C_2$
\item $R_1(X); W_1(X); R_2(X); W_2(X);  C_1; C_2$
\end{enumerate}
\vspace{1cm}
Quels sont les ordonnancements qui produisent des anomalies ?
\vspace{1cm}
\begin{enumerate}
\item $T_2$ lit une valeur de X modifiée par $T_1$ qui ne sera pas validée.$\Rightarrow$ anomalie
\item Pas d'anomalie.
\item Pas d'anomalie.
\item Problème : $W_1$ modifie une valeur qui a déjà été lue pour mise à jour.
\item Ok car équivalent à $T_1\ ;\ T_2$
\end{enumerate}
\vspace{1cm}
\section*{Cohérence ordonnancement}
\vspace{1cm}
\begin{tabular}{c|c}
$T_1$ & $T_2$\\
&\\
R(soldex); & \\
soldex = soldex+100;&\\
w(soldex);; & \\
\hline
& R(soldex); \\
&soldex = $soldex * 1.1$;\\
&W(soldex);\\
&R(soldey);\\
&W(soldey);\\
\hline
R(soldexy); & \\
soldey = soldey-100; & \\
W(soldey); & \\
C; & C ;
\end{tabular}
\vspace{1cm}

L'ordonnancement est-il cohérent ? (sérialisable) \\
soldex = 100 $\Rightarrow$ 200 $\Rightarrow$ 220\\
soldey = 100 $\Rightarrow$ 220 $\Rightarrow$ 120\\

$T_1\ ;\ T_2$ : \\
soldex = 100 $\Rightarrow$ 200 $\Rightarrow$ 220 \\
soldey = 200 $\Rightarrow$ 100 $\Rightarrow$ 110 \\
$T_2\ ; T_1$ : \\
soldex = 100 $\Rightarrow$ 110 $\Rightarrow$ 210 \\
soldey = 200 $\Rightarrow$ 220 $\Rightarrow$ 120 \\
$\Rightarrow$ Non cohérent ! $T_1\ ;\ T_2 \neq T_2\ ; T_1$  \\
\vspace{1cm}
\begin{eqnarray*}
T_1 &=& R(X); R(Y); W(X) ; W(Y) \\
T_2 &=& R(Y); W(X)
\end{eqnarray*}

\begin{enumerate}
\item[a/] Donner un ordonncement de $T_1$ et $T_2$ sérialisable
\item[b/] Donner un ordonncement de $T_1$ et $T_2$ non sérialisable
\end{enumerate}
Solution: \\
\begin{enumerate}
\item[a/] $R_1(X)$; $R_2(Y)$;\ $W_1(X)$;\ $W_2(X)$;\ $W_1(Y)$
\item[b/] $R_1(X)$;\ $R_2(Y)$; $W_2(X)$; $W_1(X)$; $W_1(Y)$
\end{enumerate}

\section*{Lock}
LS(X): Lock share de X\\
LX(X): Lock exclusive de X\\
UL(X) : unlock de X

\begin{eqnarray*}
    T_1 &=& LX(X)\ ;\ R(X)\ ;\ X=X+100;\ W(X);\ LX(Y) ;\ R(Y);\ Y=Y-100;\ W(Y);\ UL(X);\ UL(Y);\\
T_2 &=& LX(Y) ;\ R(Y) ;\ Y= Y+4 ;\ W(Y) ;\ LX(Z) ;\ Z=Z-500;\ W(Z);\ UL(Y);\ UL(Z);\\
T_3 &=& LX(Z);\ Z=0;\ W(Z);\ LX(X);\ X=X-500;\ W(X);\ UL(Z);\ UL(X)
\end{eqnarray*}
\begin{enumerate}
    \item[a)] transaction bien formés ? oui pour toutes les transactions
    \item[b)] transaction à 2 $\Phi$ ? 
\end{enumerate}
Bien formée : LS au moins pour leture, LX pour écriture et libération des verrous
2 $\Phi$ (2 phases) : Pose des verrous puis libération. Pas pose + libération + pose + libération.
(Phase d'extension puis libération).\\
Que peut-il se passer lors de l'éxécution non sequentielle de $T_1$, $T_2$ et $T_3$ ?\\
$\Rightarrow$ Risque de deadlock.
\section*{Procédure}
\begin{lstlisting}[language=Algo]
Procedure credit(solde: x, montant: y)
    debut
        LX(X);
        R(X);
        X = X+Y;
        W(X);
        UL(X);
    fin
\end{lstlisting}
\begin{lstlisting}[language=Algo]
Procedure debit(solde:x, montant y)
    debut
        LX(x);
        R(x);
        x=x-y;
        w(x);
        UL(x);
    fin
\end{lstlisting}
Ecrire la procédure Transfert.
\begin{lstlisting}
Procedure Transfert(compte: x,z, montant y)
    debut
        LX(x);
        LX(z);
        R(x);
        x=x-y;
        W(x);
        R(Z);
        z = z+y;
        W(z);
        UL(x);
        UL(z);
    fin
\end{lstlisting}
On est obligé de tout refaire ... il faut que la procédure soit à deux phases. Si on appelle les deux procédure précédente,
il y aura extension et libération deux fois.
\section*{?}
Update, delete et insert $\Rightarrow$ verouille les tuples accédés en exclusifs.\\
Commit et Rollback dévérouille tous les verrous.\\
Q1: Suffisant pour écrire des transactions bien formées et à 2 $\Phi$  ?\\
$\Rightarrow$ Non il manque le select !
\end{document}
