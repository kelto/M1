\documentclass[10pt,a4paper]{article}
\usepackage[utf8]{inputenc}
\usepackage[french]{babel}
\usepackage[T1]{fontenc}
\usepackage{amsmath}
\usepackage{amsfonts}
\usepackage{amssymb}
\usepackage{listings}
\usepackage{xcolor}

\definecolor{gris1}{gray}{0.40}
\definecolor{gris2}{gray}{0.55}
\definecolor{gris3}{gray}{0.65}
\definecolor{gris4}{gray}{0.50}
\definecolor{vert}{rgb}{0,0.4,0}
\definecolor{violet}{rgb}{0.65, 0.2, 0.65}
\definecolor{bleu1}{rgb}{0,0,0.8}
\definecolor{bleu2}{rgb}{0,0.2,0.6}
\definecolor{bleu3}{rgb}{0,0.2,0.2}
\definecolor{rouge}{HTML}{F93928}

\lstdefinelanguage{algo}{%
   morekeywords={%
    %%% couleur 1
		importer, programme, glossaire, fonction, procedure, constante, type, 
	%%% IMPORT & Co.
		si, sinon, switch, alors, fin, tantque, debut, faire, lorsque, fin lorsque, 
		declenche, declencher, enregistrement, tableau, retourne, retourner, =, pour, a,
		/=, <, >, traite,exception, 
	%%% types 
		Entier, Reel, Booleen, Caractere, Réél, Booléen, Caractère,
	%%% types 
		entree, maj, sortie,entrée,
	%%% types 
		et, ou, non,
		Debut, début, debut, fin, Fin, faux, vrai, Faux, Vrai
	},
  sensitive=true,
  morecomment=[l]{--},
  morestring=[b]',
}

\lstset{language=algo,
    %%% BOUCLE, TEST & Co.
      emph={importer, programme, glossaire, fonction, procedure, constante, type},
      emphstyle=\color{bleu2},
    %%% IMPORT & Co.  
	emph={[2]
		si, sinon, switch, alors, fin , tantque, debut, faire, lorsque, fin lorsque, 
		declencher, retourner, et, ou, non,enregistrement, retourner, retourne, 
		tableau, /=, <, =, >, traite,exception, pour, a
	},
      emphstyle=[2]\color{bleu1},
    %%% FONCTIONS NUMERIQUES
		emph={[3]Vrai, vrai, Faux, faux},
      emphstyle=[3]\color{red},
    %%% FONCTIONS NUMERIQUES
      emph={[4]entree, maj, sortie, entrée},	
      emphstyle=[4]\color{gris1},
}

\lstset{ % general style for listings 
   numbers=left 
   , literate={é}{{\'e}}1 {è}{{\`e}}1 {à}{{\`a}}1 {ê}{{\^e}}1 {É}{{\'E}}1 {ô}{{\^o}}1 {€}{{\euro}}1{°}{{$^{\circ}$}}1 {ç}{ {c}}1 {ù}{u}1
	, extendedchars=\true
   , tabsize=2 
   , stepnumber=2
   , frame=l
   , framerule=1.1pt
   , linewidth=510px
   , breaklines=true 
   , basicstyle=\footnotesize\ttfamily 
   , numberstyle=\tiny\ttfamily 
   , framexleftmargin=0mm 
   , xleftmargin=0mm 
   , captionpos=b 
	, keywordstyle=\color{bleu2}
	, commentstyle=\color{vert}
	, stringstyle=\color{rouge}
	, showstringspaces=false
	, extendedchars=true
	, mathescape=true
	, prebreak=\raisebox{0ex}[0ex][0ex]
		   {\ensuremath{\hookleftarrow}}
	,breakatwhitespace=true
} 
%	\lstlistoflistings
%	\addcontentsline{toc}{part}{List of code examples}


\title{IHM \\ TD}
\date{}
\begin{document}
\maketitle
\part*{Intro - rappel}
Liste pour construire l'IHM
\begin{enumerate}
    \item liste des événements
    \item listes des actions $\Rightarrow$ actions qui provient de l'application
    \item automate
    \item Matrice /état/événements
    \item pseudo-code
\end{enumerate}

\part*{TD1}

\section*{Question 1}
application des 4 boutons: voir cours. \\

\begin{enumerate}
    \item liste des event: Bouton1, Bouton2, Bouton3, Bouton4
    \item liste des actions: (Activation/Desactivation des boutons ne fait pas partie des actions). Il n'y en a donc pas.
    \item automate: automate1.dia
    \item Matrice état/événements
\end{enumerate}

\begin{tabular}{c|c|c|c|c}
\hline
Etat-event & Btn1 & Btn2 & Btn3 & Btn4 \\
\hline
Init & S = E2 & interdit & interdit & interdit\\
\hline
E2 & interdit & S=E3 => activateE3() & interdit &interdit\\
\hline
E3 & interdit & interdit & S=E4 => activateE4() & interdit\\
\hline
E4 & interdit & interdit & interdit & S=INIT=>activiateInit()\\ \end{tabular}

pseudo-code:
\begin{lstlisting}[language=C]
EventHandlerCb1
switch(s)
{
    case INIT:
        S = Etat2;
        activateEtat2();
        break;
    case Etat2:
        //interdit
    case Etat3:
        //interdit
    case Etat4:
        //interdit
}
\end{lstlisting}

\section*{Question 2}

\subsection*{Liste des event}
clique bouton1, bouton2, bouton3, bouton4.

\subsection*{Liste des actions}
Aucunes actions.

\subsection*{Automate}

\begin{tabular}{c|c|c|c|c}
\hline
Etat-event & btn1 & btn2 & btn3 & btn4\\
\hline
Init & S=E2 & S=E3 & interdit & interdit\\
\hline
E2 & S=E2 & S=E4 & interdit & interdit \\
\hline
E3 & S=E4 & S=E3 & interdit & interdit \\
\hline
E4 & interdit & interdit & S=E5 & S=E6\\
\hline
E5 & interdit & interdit & S=E5 & S=Init\\
\hline
E6 & interdit & interdit & S=Init & S=E6 \\
\end{tabular}

\begin{lstlisting}[language=C]

EventHandlerCb3()
{
    switch(s)
    {
        case INIT:
            //interdit
        break;
        case E2:
            //interdit
        break;
        case E3:
            //interdit
        break;
        case E4:
            S=E6;
            activateE6();
        break;
        case E5:
            S=INIT;
            activateINIT();
        break;
        case E6:
            S=E6;
            activateE6();
        break;
    }
}

activateE6 ()
{
    JButton1.setEnable(False);
    JButton2.setEnable(False);
    JButton3.setEnable(True);
    JButton4.setEnable(True);
}
\end{lstlisting}

\section*{Question bonus (par le prof)}

* liste des event : stop, start, tic\\
* liste des actions :
\begin{enumerate}
    \item afficheSoleil()
    \item afficheNuage()
    \item affichePluie()
    \item afficheEclair()
    \item affichePouf()
    \item afficheVide()
\end{enumerate}
*automate : automate3.dia

\begin{tabular}{c|c|c|c}
\hline
Etat-Event & stop & start & tic \\
\hline
Init & interdit & S=E1 & interdit \\
\hline
E1 & S=E5 & interdit & S=E2 \\
\hline
E2 & S=E5 & interdit & S=E3 \\
\hline
E3 & S=E5 & interdit & S=E4 \\
\hline
E4 & S=E5 & interdit & S=Init \\
\hline
E5 & interdit & S=E1 & interdit

\end{tabular}


\end{document}
