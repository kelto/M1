\documentclass[10pt,a4paper]{article}
\usepackage[utf8]{inputenc}
\usepackage[french]{babel}
\usepackage[T1]{fontenc}
\usepackage{amsmath}
\usepackage{amsfonts}
\usepackage{amssymb}
\usepackage{listings}
\usepackage{xcolor}

\title{Architecture Programmation Parallele\\ TD}
\date{}
\definecolor{gris1}{gray}{0.40}
\definecolor{gris2}{gray}{0.55}
\definecolor{gris3}{gray}{0.65}
\definecolor{gris4}{gray}{0.50}
\definecolor{vert}{rgb}{0,0.4,0}
\definecolor{violet}{rgb}{0.65, 0.2, 0.65}
\definecolor{bleu1}{rgb}{0,0,0.8}
\definecolor{bleu2}{rgb}{0,0.2,0.6}
\definecolor{bleu3}{rgb}{0,0.2,0.2}
\definecolor{rouge}{HTML}{F93928}
\lstset{ % general style for listings 
    numbers=left 
    , literate={é}{{\'e}}1 {è}{{\`e}}1 {à}{{\`a}}1 {ê}{{\^e}}1 {É}{{\'E}}1 {ô}{{\^o}}1 {€}{{\euro}}1{°}{{$^{\circ}$}}1 {ç}{ {c}}1 {ù}{u}1
    , extendedchars=\true
    , tabsize=2 
    , stepnumber=2
    , frame=l
    , framerule=1.1pt
    , linewidth=510px
    , breaklines=true 
    , basicstyle=\footnotesize\ttfamily 
    , numberstyle=\tiny\ttfamily 
    , framexleftmargin=0mm 
    , xleftmargin=0mm 
    , captionpos=b 
    , keywordstyle=\color{bleu2}
    , commentstyle=\color{vert}
    , stringstyle=\color{rouge}
    , showstringspaces=false
    , extendedchars=true
    , mathescape=true
    , prebreak=\raisebox{0ex}[0ex][0ex]
    {\ensuremath{\hookleftarrow}}
    ,breakatwhitespace=true
} 
%	\lstlistoflistings
%	\addcontentsline{toc}{part}{List of code examples}


\begin{document}
\maketitle

\part*{TD2}

\begin{lstlisting}[language=C]
// cree une equipe de threads
#pragma omp parallel 

// variable d'environnement permettant de definir le nombre de threads pour une equipe.
OMP_NUM_THREADS 

// meme chose mais avec une fonction
omp_set_num_threads(int nb)

// creer une equipe de 8 threads
#pragma omp parallel num_threads(8)


// parallelisation de la boucle for
// avec 4 tours de boucles pour chaque threads.
#pragma omp for schedule(static,4)
for(....)
{

} 

// allocation static plus efficace !
// n'importe quel thread peut faire n'import quel paquet de 4
// explication de merde du prof ...
#pragma omp for schedule(dynamic,4)
for(;;)
{

}

// execute uniquement par un thread
#pragma omp single
(
}

// pour faire une barriere explicite
#pragma omp barrier
\end{lstlisting}


\section*{Exercice 1}

\begin{lstlisting}[language=C]

void saxpy(float * array,int size, const float a, const float y)
{
    // on devrait s'occuper du nombre de threads pour que cela tombe juste avec la taille du tableau
    // ce n'est pas fait ici
    #pragma omp parallel num_threads(8) private(i) shared(a,array,y)
    {
        #pragma omp for 
        for(int i = 0; i<size; i++)
        {
            array[i] = (array[i] * a)+y;
        }
    }
}

\end{lstlisting}

\section*{Exercice 2}

On vérifie que les boucles sont parallélisable. On voit que oui, donc ...

\begin{lstlisting}[language=C]

#prama omg parallel private(i,j) shared(colterm, rowterm, a, b,m,q,p)
{
    #pragma omp for
    for(i=0; i<m; i++)
    {
        rowterm[i] = 0.0 ;
        for(j=0; j<p; j++)
        {
            rowterm[i] += a[i][2*j] * a[i][2*j+1];
        }
    }

    #pragma omp for
    for(i=0; i < q ; i++)
    {
        colterm[i] = 0.0;
        for(j = 0; j < p ; j++)
        colterm[i] += b[2*j][i] * b[2*j+1][i];
    }
}

\end{lstlisting}

\section*{Exercice 3}

\begin{lstlisting}[language=C]

float temp[N] = {100.0, 0.0,0.0,...,0.0};
int t, s;
/* calcul de l'evolution des temperatures */
#pragma omp parallel private(s) shared(temp, t)
{
    //on prend le nombre de threads
    p = omp_get_num_threads;
    for(t = 0; t < TMAX ; t++ )
    {
        #pragma omp for schedule((N-2)/p)
        for (s = 1; s < N-1 ; s++)
        temp[s] = (temp[s-1] + temps[s+1])/2.0 ;
    }
}

\end{lstlisting}

\section*{Exercice 4}
\subsection*{Question 1}
De base non le tri à bulle ne semble pas parallélisable.

\subsection*{Question 2}
Possible car on lance les thread à chaque étapes, ils ne se font pas chier les uns les autres comme ça.
\begin{lstlisting}[language=C]

#pragma omp parallel private(i,temp)
{

    for(step = N; step > 0 ; step--)
    {
        if(step%2 == 0)
        {
            #pragma omp for
            for(i=0; i < N-1 ; i+=2)
            {
                if(Tab[i]>Tab[i+1]
                {
                    temp = Tab[i];
                    Tab[i] = Tab[i+1];
                    Tab[i+1] = temp;
                }
            }
        }
        else
        {
            #pragma omp for
            for(i=1; i < N-1 ; i+=2)
            {
                if(Tab[i]>Tab[i+1)
                    {
                        temp = Tab[i];
                        Tab[i] = Tab[i+1];
                        Tab[i+1] = temp;

                    }
                }
            }
        }

    }

    \end{lstlisting}

    \section*{Exercice 5}

    \subsection*{Question 1}

    \begin{lstlisting}
    #pragma omp parallel private()
    {
        tid = omp_get_thread_num();
        p = om_get_thread_num();
        somme[tid] = 0;
        #pragma omp for schedule(static, n/p)
        for(i=0; i<n; i++)
        {
            somme[tid] = somme[tid] + v[i];
        }
        #pragma omp single
        sommeTotal = 0;
        for(i=0; i<p;i++)
        {
            sommeTotal += somme[i];	
        }


        \end{lstlisting}

        \subsection*{Question 2}

        \begin{lstlisting}[language=C]
        #pragma omp parallel private(tid ,i ,v ,step , voisin) shared(n ,p ,somme ,globalsomme)

        p = omp_get_num_threads();
        tid = omg_get_thread_num();
        somme[tid] = 0;

        #pragma omp for schedule(static, n/p)
        for(i=0; i < n; i++)
        {
            somme[tid] += v[i];
        }
        for(step = 1; step <= p ; step*=2)
        {
            voisin = somme[(tid+step)%p];
                #pragma omp barrier;
                somme[tid] += voisin;
            }
            \end{lstlisting}

            \part*{TD 3}

            \section*{Exercice 1}

            \
            Problème section critique :
            \begin{lstlisting}
            #pragma omp critical <name>
            {

            }

            \#pragma omp atomic
            {

            }

            accés atomique :
            compare and swap
            do {
                old_count = count;
                newcount = old_count+1;
                success = compareandswap(&count, old_count, new_count);
            }while(!success)
            \end{lstlisting}

            compareandswap: si ce qu'il y a à l'addresse $\&count$ est égale à $old\_count$, on écrit $new\_count$ à cette addresse.


            \subsection*{Question 1}

            \begin{lstlisting}[language=C]
            #pragma omp parallel private(i,j) shared (A,B,moyenne,somme)
            int A[N][P], B[N][P];
            int somme = 0;
            float moyenne;
            #pragma omp for
            for(int i = 0; i < N; i++)
            for(int j = 0; j < N; j++)
            #pragma omp atomic
            somme += A[i][j];

            #pragma omp single
            moyenne = somme /(n*p);
            #pragma omp for
            for(i = 0; i < N; i++)
            for(j=0; j<P; j++)
            if(A[i][j] >= moyenne)
            B[i][j] = 1
            else
            B[i][j] = 0;

            \end{lstlisting}

            Question b) :
            \begin{lstlisting}[language=C]
            #pragma omp parallel private(i,j) shared (A,B,moyenne,somme)
            int A[N][P], B[N][P];
            int somme = 0;
            float moyenne;
            #pragma omp for reduction(+: somme)
            for(int i = 0; i < N; i++)
                for(int j = 0; j < N; j++)
                    somme += A[i][j];

            #pragma omp single
            moyenne = somme /(n*p);
            #pragma omp for
            for(i = 0; i < N; i++)
                for(j=0; j<P; j++)
                    if(A[i][j] >= moyenne)
                        B[i][j] = 1
                    else
                        B[i][j] = 0;

            \end{lstlisting}

            \section{Exerice 2}

            \begin{lstlisting}[language=C]
            void calculer(float *A){
                int i,j, fini = 0;
                float diff = 0, temp;
                while(!fini)
                {
                    diff = 0;
                    for(i=0; i<n; i++)
                    {
                        for (j=0; j<n; j++)
                        {
                            temp = A[i][j];
                            A[i][j] = 0.2 * (A[i][j] + A[i][j-1] + A[i-1][j]
                            + A[i][j+1] + A[i+1][j]);
                            diff = diff + abs(A[i][j] - temp);
                        }
                    }
                    if(diff / (n*n) < SEUIL)
                    fini = 1;
                }
            }
            solution:

            \begin{lstlisting}[language=C]
            void calculer(float *A){
                #pragma omp parallel private(i, j, temp) shared(A, diff, SEUIL, n, fini)
                int i,j, fini = 0;
                float diff = 0, temp;
                while(!fini)
                {
                    diff = 0;
                    #pragma omp for reduction(+: diff)
                    for(i=0; i<n; i++)
                    {
                        for (j=0; j<n; j++)
                        {
                            temp = A[i][j];
                            A[i][j] = 0.2 * (A[i][j] + A[i][j-1] + A[i-1][j]
                            + A[i][j+1] + A[i+1][j]);
                            diff = diff + abs(A[i][j] - temp);
                        }
                    }
                    if(diff / (n*n) < SEUIL)
                    fini = 1;
                }
            }
            \end{lstlisting}
            Comme dans l'exercice avec les temperatures, on suppose que ça converge, donc pas de soucis
            si c'est pas vraiment parallelisable.
\section*{Exercice 3}
\begin{lstlisting}[language=C]
            int max = 0,i; 
            unsigned int A[N];
            
            #pragma omp parallel for
            for(i = 0; i < N; i++)
            {
                // pour eviter de passer par critical à chaque fois
                // ce qui ralentirait beaucoup la parallélisation.
                if(A[i] > max)
                {
                    #pragma omp critical{
                        if(A[i] > max)
                            max = A[i];
                    }
                }
            }
\end{lstlisting}

\section*{Exercice 4}
\begin{lstlisting}[language=C]
int s, i, j;
for(s=0; s < NUM_STEPS ; s++)
{
    for(i=0; i<N; i++)
    {
        strength[i] = 0;
        for(j=0; j<N; j++)
        {
            strenght[i] += interaction(i,j);
        }
    }
    for(i=0; i<N; i++)
    {
        update(i);
    }
}
\end{lstlisting}

\section*{Question 1}

\begin{lstlisting}
int s, i, j;
#pragma omp parallel private(s,i,j, temp)
for(s=0; s < NUM_STEPS ; s++)
{

    #pragma omp for
    for(i=0; i<N; i++)
    {
        strength[i] = 0;
        for(j=0; j<N; j++)
        {
            strenght[i] += temp;
        }
    }
    #pragma omp for
    for(i=0; i<N; i++)
    {
        update(i);
    }
}
\end{lstlisting}

\section*{Question 2}

\begin{lstlisting}[language=C]
int s, i, j;
#pragma omp parallel private(s,i,j, temp)
for(s=0; s < NUM_STEPS ; s++)
{
    #pragma omp for
    for(i=0; i<N; i++)
        strength[i] = 0;

        #pragma omp for schedule(dynamic,1)
    for(i=0; i<N; i++)
    {
        // On ne parcours que jusqu'à i cette fois.
        for(j=0; j<i; j++)
        {
            temp = interaction(i,j);

            #pragma omg atomic
            strenght[i] += temp;

            #pragma omg atomic
            strenght[j] -= temp;
        }
    }
    #pragma omp for
    for(i=0; i<N; i++)
    {
        update(i);
    }
}
\end{lstlisting}

Dans cette versions, les threads n'ont pas la même charge de travail, les premiers n'ont quasiment aucun travail.\\

\section*{Question 3}

\begin{lstlisting}[language=C]
int s, i, j;
#pragma omp parallel private(s,i,j, temp)
for(s=0; s < NUM_STEPS ; s++)
{
    #pragma omp for
    for(i=0; i<N; i++)
        strength[i] = 0;

    #pragma omp for
    for(i=0; i<N; i++)
    {
        // On ne parcours que jusqu'à i cette fois.
        for(j=0; j<i; j++)
        {
            temp = interaction(i,j);

            #pragma omg atomic
            strenght[i] += temp;

            #pragma omg atomic
            strenght[j] -= temp;
        }
    }
    #pragma omp for
    for(i=0; i<N; i++)
    {
        update(i);
    }
}

\part*{TD 4}

Voir l'écriture différée avec le cache.
\section*{Exercice 1}

\subsection*{Question 1}
Réponse : Etat1.dia
(voir sujet TD 4, expliqué pour les conventions)
Event/resultatAppel[Action] (pour les transitions)

voir correspondance direct, cache et bus.

\end{document}
