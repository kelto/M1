\documentclass[10pt,a4paper]{article}
\usepackage[utf8]{inputenc}
\usepackage[french]{babel}
\usepackage[T1]{fontenc}
\usepackage{amsmath}
\usepackage{amsfonts}
\usepackage{amssymb}
\usepackage{listings}
\usepackage{xcolor}

\definecolor{gris1}{gray}{0.40}
\definecolor{gris2}{gray}{0.55}
\definecolor{gris3}{gray}{0.65}
\definecolor{gris4}{gray}{0.50}
\definecolor{vert}{rgb}{0,0.4,0}
\definecolor{violet}{rgb}{0.65, 0.2, 0.65}
\definecolor{bleu1}{rgb}{0,0,0.8}
\definecolor{bleu2}{rgb}{0,0.2,0.6}
\definecolor{bleu3}{rgb}{0,0.2,0.2}
\definecolor{rouge}{HTML}{F93928}

\lstdefinelanguage{algo}{%
   morekeywords={%
    %%% couleur 1
		importer, programme, glossaire, fonction, procedure, constante, type, 
	%%% IMPORT & Co.
		si, sinon, switch, alors, fin, tantque, debut, faire, lorsque, fin lorsque, 
		declenche, declencher, enregistrement, tableau, retourne, retourner, =, pour, a,
		/=, <, >, traite,exception, 
	%%% types 
		Entier, Reel, Booleen, Caractere, Réél, Booléen, Caractère,
	%%% types 
		entree, maj, sortie,entrée,
	%%% types 
		et, ou, non,
		Debut, début, debut, fin, Fin, faux, vrai, Faux, Vrai
	},
  sensitive=true,
  morecomment=[l]{--},
  morestring=[b]',
}

\lstset{language=algo,
    %%% BOUCLE, TEST & Co.
      emph={importer, programme, glossaire, fonction, procedure, constante, type},
      emphstyle=\color{bleu2},
    %%% IMPORT & Co.  
	emph={[2]
		si, sinon, switch, alors, fin , tantque, debut, faire, lorsque, fin lorsque, 
		declencher, retourner, et, ou, non,enregistrement, retourner, retourne, 
		tableau, /=, <, =, >, traite,exception, pour, a
	},
      emphstyle=[2]\color{bleu1},
    %%% FONCTIONS NUMERIQUES
		emph={[3]Vrai, vrai, Faux, faux},
      emphstyle=[3]\color{red},
    %%% FONCTIONS NUMERIQUES
      emph={[4]entree, maj, sortie, entrée},	
      emphstyle=[4]\color{gris1},
}

\lstset{ % general style for listings 
   numbers=left 
   , literate={é}{{\'e}}1 {è}{{\`e}}1 {à}{{\`a}}1 {ê}{{\^e}}1 {É}{{\'E}}1 {ô}{{\^o}}1 {€}{{\euro}}1{°}{{$^{\circ}$}}1 {ç}{ {c}}1 {ù}{u}1
	, extendedchars=\true
   , tabsize=2 
   , stepnumber=2
   , frame=l
   , framerule=1.1pt
   , linewidth=510px
   , breaklines=true 
   , basicstyle=\footnotesize\ttfamily 
   , numberstyle=\tiny\ttfamily 
   , framexleftmargin=0mm 
   , xleftmargin=0mm 
   , captionpos=b 
	, keywordstyle=\color{bleu2}
	, commentstyle=\color{vert}
	, stringstyle=\color{rouge}
	, showstringspaces=false
	, extendedchars=true
	, mathescape=true
	, prebreak=\raisebox{0ex}[0ex][0ex]
		   {\ensuremath{\hookleftarrow}}
	,breakatwhitespace=true
} 
%	\lstlistoflistings
%	\addcontentsline{toc}{part}{List of code examples}

\author{Charles Fallourd}
\title{Traduction des Langages\\ TD}
\date{}
\begin{document}
\maketitle

\part*{TD 5 : Quadruplet}


\begin{flalign*}
<statlist> &:= <inst><suite>\\
<suite> &:= ; <statlist> | \lambda\\
<inst> &:= ... | <ifstat> | <whilestat> | <repeatstat> | <forstat> | <casestat> | ... \\
<repeatstat> &:= repeat <statlist> until <exp> endrepeat \\
<forstat> &:= for ident := <exp1> step <exp2> until <exp3> do <statlist> endfor \\
<casestat> &:= case <exp> of <alternative> { ; <alternative> }* \\
&otherwise <statlist> endcase\\
\end{flalign*}



\section*{Methodologie pour les trois questions}
\begin{enumerate}
\item Donner le schéma équivalent en quadruplets.
\item Inventaire des "problèmes" : verification types et gestion des adresses des quadruplets. Ainsi que l'inventaire des solutions apportées.
\item Mise en oeuvre des procédures de descente récursive pour engendrer le schéma en quadruplets voulus en tenant compte des solutions apportées.
\end{enumerate}

\section*{Exercice 1}
Mettre en oeuvre un processus de traduction en quadruplets des structures de controle pour <repeatstat>.

\subsection*{Schéma équivalent en quadruplet}

$\\
\uparrow \\
\vdots \ \ quadruplet\  engendre\ par\ <statlist>\\
\downarrow\\
\uparrow \\
\vdots \ \ quadruplet\  engendre\ par\ <exp>\\
\downarrow\\
=?, <exp>.res, 0, DEBUT
$
ici la dernière ligne est un quadruplet, (=?, exp.res, 0, DEBUT)

\subsection*{Inventaire des problèmes et solution apportées}

\begin{enumerate}
\item Verif Type : verifier que le resultlist de EXP(<exp>.res) est une variable déclarée Booléenne. 
$\Rightarrow$ CHECKTYPE
\item Référence en arrière à DEBUT
$\Rightarrow$ variable DEBUT qui contiendra l'@ du 1\ier{} quadruplet engendré par STATLIST qui est dans NEXTQUAD
\end{enumerate}

\subsection*{Mise en oeuvre }
\begin{lstlisting}[language=Algo, framerule=0pt,  numbers=none]
procedure REPEATSTAT is
	Res: String; -- nom de la variable qui contiendra le resultat du test
	Debut: Integer; -- garder l'@ du $1\ier{}$ quadruplet engendre par STATLIST
	
	begin
		SKIP('repeat');
		Debut := NEXTQUAD;
		STATLIST ;
		SKIP('until');
		EXP(Res);
		CHECKTYPE(Res, Boolean);
		GEN("=?," ^Res^," O,"^str(Debut));
		SKIP('endrepeat');
	end REPEATSTAT;
\end{lstlisting}

\section*{Exercice 2}
De meme pour <forstat>

\subsection*{Schéma équivalent en quadruplet}

$\\
\uparrow \\
\vdots \ \ quadruplet\  engendre\ par\ <exp2>\\
\downarrow\\
:= , <exp1>.res, nil, ident
\uparrow \\
\vdots \ \ quadruplet\  engendre\ par\ <exp2>\\
\downarrow\\
\uparrow \\
\vdots \ \ quadruplet\  engendre\ par\ <exp3>\\
\downarrow\\
TEST > ?, ident, <exp3>.res, FIN \\
\uparrow \\
\vdots \ \ quadruplet\  engendre\ par\ <statlist>\\
\downarrow\\
+, ident, <exp2>.res, ident\\
goto, nil, nil\ TEST
FIN
$

ici TEST (> ?), (ident), (<exp3>.res), (FIN ), TEST est un label, et c'est un quadruplet jump conditionnel. Si (> ?) alors jump FIN.

\subsection*{Inventaire des problèmes et solution apportées}

\begin{enumerate}
\item verif type : les 3 résultats $res_i$ des 3 <$exp_i$> doivent être des variables déclarées \emph{Integer} (prob 1)
\item Référence en avant à FIN : \subitem engendrer un quadruplet incomplet (4\ieme{} champ à nil) (prob 2.)
\subitem conserver l'@ de ce quadruplet incomplet dans une variable TEST (2.1)
\subitem mettre à jour le 4\ieme{} champs (@) du quadruplet incomplet quand on connait l'@ FIN $\Rightarrow $BACKPATCH (2.2)
\item Référence en arrière à TEST $\Rightarrow$ conserver l'@ du quadruplet TEST (2.3)
\end{enumerate}

\subsection*{Mise en oeuvre}

\begin{lstlisting}[language=Algo, framerule=0pt,  numbers=none]

procedure FORSTAT is
	Res1, Res2, Res3 : String -- resultats des 3 <$exp_i$>
	Test : Integer;
	VarBoucle : String;
	begin
		SKIP('for');
        if NEXTS != 'ident' then ERREUR; endif; -- 'ident' = unite lexical 'ident' (genre que c'est une variable)
		varBoucle := NEXTS.Valeur ; -- ici on prend la valeur de l'unite lexical ident
		SCAN;
		SKIP(':=');
		EXP(Res1);
		CHECKTYPE(Res1, Integer); -- (prob 1)
		GEN(":=,"^Res1^",nil,"^varBoucle);
		SKIP('step');
		EXP(Res2);
		CHECKTYPE(Res2, Integer); -- (prob 1)
		SKIP('until');
		EXP(Res3);
		CHECKTYPE(Res3, Integer); -- (prob 1)
		Test := NEXTQUAD; -- (prob 2.2)
		GEN(">?,"^VarBoucle^","^Res3^",nil"); -- (prob 2.1)
		SKIP('do');
		STATLIST;
		GEN("+,"^VarBoucle^","^Res2^","^VarBoucle);
		GEN("goto,nil,nil"^str(teST));
		BACKPATCH([Test], NEXTQUAD); -- (prob 3)
		SKIP('endfor');
	end FORSTAT ;
		

\end{lstlisting}
$\bigwedge$ est l'opérateur de concaténation !
\newpage
\section*{Exercice 3}
De meme pour <casestat>
\subsection*{Schéma équivalent en quadruplet}
$\\
\uparrow \\
\vdots  \ quadruplet\ engendrer\ par\ <exp> \\
\downarrow \\
\neq ?\ , \ <exp>.res,\ nb1,\ ALT2\\
\uparrow \\
\vdots  \ quadruplet\ engendrer\ par\ <alternative>  pour S1\\
\downarrow \\
goto, nil, nill,\ FIN\\
ALT2\ \neq ?\ , \ <exp>.res,\ nb1,\ ALT3\\
\uparrow \\
\vdots  \ quadruplet\ engendrer\ par\ <alternative>  pour S2\\
\downarrow \\
goto, nil, nill,\ FIN\\
\vdots \ ALT3 \\
ALT_p \ \neq ? <exp>.res,\ nb_p,\ OTHER\\
\vdots  \ quadruplet\ engendrer\ par\ <alternative>  pour Sp\\
goto, nil, nill,\ FIN\\
OTHER\\
\uparrow \\
\vdots  \ quadruplet\ pour\ S_{p+1} \\
\downarrow \\
FIN\\
$


\subsection*{Inventaire des problèmes et solution apportées}
\begin{enumerate}
    \item Verif type: Integer de <exp>.res $\Rightarrow$ CHECKTYPE $\alpha$
    \item Reference en avant à $ALT_{i+1}$ \\
        mémoriser dans une variable ALT ($\beta_1$) l'adresse du quadruplet engendré\\
        incomplet et maj du champs @ de ce quadruplet quand on connait. $\beta_2$\\
        $Rightarrow$ $\beta_3$ $\Rightarrow$ BACKPATCH
    \item Réference en avant à FIN\\
        mémoriser les @ des quadruplet engendrés $(\gamma_1)$ incompletdans une liste $(\gamma_2)$\\
        et on met à jours les champs @ de tous les audruplet de la liste avec la même @ FIN quand on la connait\\
        $\Rightarrow $ BACKPATCH $\gamma_3$
        \item Verification "sémantique"\\
            $\Rightarrow$ chaque $nb_i$ doit être différent. On doit donc avoir une liste de nombre.\\
            on suppose qu'on a un outil MEMBER(E,L) $= VRAI\ si\ e\ \epsilon\ L$ 

\end{enumerate}
\subsection*{Mise en oeuvre}
\begin{lstlisting}[language=Algo]
procedure CASESTAT is
    Res: String ;-- resultatde EXP
    ListFin: list of Integer;
    ListNbre: list of Integer;
    begin 
        SKIP('case');
        EXP(Res);
        CHECKTYPE(Res,Integer); -- $\alpha$
        SKIP('of');
        ListFin: MAKELIST;
        ListNbre: MAKELIST; -- on creer les listes vide
        ALTERNATIVE(Res,ListNbre);
        -- NEXTQUAD -1 contient l'@ du 
        -- goto, nil,nil,nil
        ListFin := MERGE(ListFin, [NEXTQUAD-1]); -- $\gamma_1$
        While NEXTS = ';' loop
            SKIP(';'); -- ou SCAN puisqu'on sait déjà que c'est un ';'
            ALTERNATIVE(Res,ListNbre);
            ListFin := MERGE(ListFin, [NEXTQUAD-1]);
        endloop;
        SKIP('otherwise');
        STATLIST;
        BACKPATCH(ListFin,NEXTQUAD); -- $\gamma_3$
        SKIP('endcase');
    end CASESTAT;
\end{lstlisting}
\begin{lstlisting}
procedure ALTERNATIVE(Res in String, ListNbre in out) is
    ALT: Integer;
    begin
        if NEXTS $\neq$ 'number' then ERREUR; endif;
        if MEMBER(NEXTS.Valeur, ListNbre) then ERREUR; endif; -- $\alpha$
        ListNbre := MERGE([NEXTS.Valeur], ListNbre);
        ALT := NEXTQUAD; -- $\beta_1$
        GEN("$\neq$ ?,"^Res^","^str(NEXTS.Valeur)^",nil"); -- $\beta_2$
        SCAN;
        SKIP(':');
        STATLIST;
        GEN("goto,nil,nil,nil"); -- $\gamma_2$
        BACKPATCH([ALT], NEXTQUAD); -- $\beta_3$
    end ALTERNATIVE;

\end{lstlisting}

\part*{TD 6  \\ Génération de code \& analyse ascendante}

Todo : 
    * action sémantique = ?

\section*{Méthode de la descente récursive}

\begin{flalign*}
<statlist> &:= <inst><suite>\\
<suite> &:= ; <statlist> | \lambda\\
<inst> &:= ... | <ifstat> | <whilestat> | <repeatstat> | <forstat> | <casestat> | ... \\
<loopstat> &:= ident: loop <statlist> endloop\\
<exitstat> &:= when <exp> exit ident
\end{flalign*}


\begin{lstlisting}[language=Algo]
A: loop
    ...
    when E1 exit A;
    ...
    B : loop
        ...
        when E2 exit A;
        ...
        when E3 exit B;
        ...
        endloop;
    ...
    when E4 exit A;
    ...
    endloop
\end{lstlisting}

\subsection*{Schéma équivalent en quadruplet de l'exemple}
$\\
Début A
\uparrow \\
\vdots  ... (quelque chose ...)\\
\downarrow \\
\uparrow \\
\vdots  \ quadruplet\ engendré\ pour\ E_{1} \\
\downarrow \\
\neq ?\ , \ E1.res,\ 1,\ finA Il\ s'agit\ de\ <exitstat>\\
\uparrow \\
\vdots  ... (quelque chose ...)\\
\downarrow \\
Début\ B\\
\uparrow \\
\vdots  ... (quelque chose ...)\\
\downarrow \\
\uparrow \\
\vdots  \ quadruplet\ pour\ E_{2} \\
\downarrow \\
\uparrow \\
\downarrow \\
\neq ?\ , \ E2.res,\ 1,\ finA Il\ s'agit\ de\ <exitstat>\\
\uparrow \\
\vdots  ... (quelque chose ...)\\
\downarrow \\
\uparrow \\
\vdots  \ quadruplet\ pour\ E_{3} \\
\downarrow \\
\neq ?\ , \ E3.res,\ 1,\ finB Il\ s'agit\ de\ <exitstat>\\
\uparrow \\
\vdots  ... (quelque chose ...)\\
\downarrow \\
goto, nil,nil,'DebutB'\\
finB\\
\uparrow \\
\vdots  ... (quelque chose ...)\\
\downarrow \\
\uparrow \\
\vdots  \ quadruplet\ pour\ E_{4} \\
\downarrow \\
\neq ?\ , \ E4.res,\ 1,\ finA Il\ s'agit\ de\ <exitstat>\\
\uparrow \\
\vdots  ... (quelque chose ...)\\
\downarrow \\
goto,nil,nil,DebutA\\
finA\\
$

\subsection*{Inventaire des problèmes et solution apportées}
\begin{enumerate}
    \item Vérifier le type booléen des $E_{i}$.res (des exitstat).
        $\Rightarrow$ CHEKTYPE.
    \item Référence en arrière à Début $\Rightarrow$ mémoriser l'@ dans une variable.
    \item Référence en avant à Fin $\Rightarrow$ engendrer un quad incomplet,
    mémoriser son @ dans liste de quad incomplets et on met à jour le champs
    @ avec BACKPATCH. 
    $\Rightarrow$ Associer la liste de quad incomplets à un nom de boucle.
\end{enumerate}

Attention !!
On ne veut pas de boucles imbriquées de même nom.\\
On peut vouloir le même nom sur deux boucles "consécutives".\\
$\Rightarrow$ On va utiliser la tables des symboles TDS.
Dans l'exemple suivant on pourrait rajouter d'autres informations.\\

lqi = liste des quadruplets incomplets.\\
\begin{tabular}{|c|c|c|c|c|c|}
    \hline
    & nomBoucle  & lqi & type&&\\
    \hline
    1 & &&&&\\
    \hline
    P & A & TDS[P].lqi& loop &&\\
    \hline
    &&&&&\\
    \hline
    Q & B &&&&\\
\end{tabular}

On va vérifier que le nom d'une boucle imbriqué n'est pas déjà dans la table \\
$\Rightarrow$ SEARCH(S:string, out p:int, T:type) . Si p = 0, n'existe pas dans TDS.
nom de boucle et même type boucle différent ! Si c'est un autre type, pas grave.
Pour ajouter à TDS $\Rightarrow$  ENTER(S:String, out P:int)\\
CANCEL(S:String, T:type) $\Rightarrow$ va supprimer l'entré de type T et de nom S.

\begin{lstlisting}[language=Algo]
procedure LOOPSTAT(in S:String) 
    -- S est le nom de la boucle (voir ident ... apparemment la meme chose)
    P: Integer;
    DEBUT: Integer;
    begin
        SKIP('loop');
        SEARCH(S,P, 'loop');
        si P!=0 then ERREUR;
        -- on a vérifié que le nom de boucle n'est pas
        -- déjà utilisé comme boucle englobante

        -- on met le nom S dans TDS
        ENTER(S,P);
        TDS[P].types := 'loop';
        TDS.[P].lqi := MAKELIST;

        -- on mémorise l'@ du premier quad DEBUT
        DEBUT := NEXTQUAD;
        STATLIST;
        SKIP('endloop');
        GEN("goto,nill,nill"^str(DEBUT));
        BACKPATCH(TDS[P].lqi,NEXTQUAD);

        -- effacer le nom de la boucle de la TPS
        CANCEL(S,'loop');
    end LOOPSTAT

\end{lstlisting}

\begin{lstlisting}[language=Algo]
procedure EXITSTAT is
    Res: String;
    P: Integer;
    begin
    SKIP('when');
    EXP(RES);
    CHECKTYPE(Res, Boolean);
    SKIP('exit');
    if NEXTS != 'identicateur' then ERREUR;
    SEARCH(NEXTS.String, P, 'loop');
    if P = 0 then ERREUR;
    TDS[P].lqi := MERGE(TDS[P].lqi,[NEXTQUAD]);
    GEN("=?,"^Res^",s,nill");
    SCAN;
    end EXITSTAT

\end{lstlisting}

\section*{Traduction par analyse ascendante}


\begin{flalign}
<forstat> &:= ident:= <exp1>step<exp2>until<exp3>do<statlist>endfor\\
<loopstat> &:= ident: loop <statlist> endloop
\end{flalign}

\subsection*{Question 1}

$\Rightarrow$ analyseur LR(k). le non terminal le plus à Droite.\\
k=1 dans la réalité.\\
Ces analyseur trouvent une grammaire et la réduit.\\
On part du mot et on remonte vers l'axiome.\\
A se dérive en B et B se réduit en A. (ouai d'un coup c'est plus clair ><)\\

Pour se faire:\\
même schéma en quadruplet,\\
même problemes et solutions.\\
MAIS la mise en oeuvre est différente. Ce n'est plus des procédures de descente récursive.\\
A la place:
\begin{enumerate}
    \item coupures : où ?
    \item piles sémantiques ? conserver des noms. Conserver les @ des quadruplets.
\end{enumerate}
Piles sémantiques : Pile des noms Res. pile des adresses des quadruplet Adr.\\
Pour accéder à l'info : nonTerminal.NomPile\\

\subsection*{Mise en oeuvre en ascendant}
\begin{lstlisting}
SEARCH(ident.Res,P,'loop');
if P!=0 then ERREUR;
ENTER(ident.Res,P);
TDS[P].lqi := MAKELIST;
TDS[P].type := 'loop';
M.Adr := NEXTQUAND;
STATLIST;
\end{lstlisting}
Alors ça, aucune idée d'où ça va ...
\begin{lstlisting}
GEN("goto,nil,nil,"^str(M.Adr));
SEARCH(ident.Res,P,'loop');
BACKPATCH(TDS[P].lqi,NEXTQUAD);
\end{lstlisting}
\end{document}
